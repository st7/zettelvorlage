\documentclass[fontsize=11pt,paper=a4,final]{scrartcl} %draft
\usepackage[utf8]{inputenc}
\usepackage[ngerman]{babel}
\usepackage[T1]{fontenc}
\usepackage{textcomp}
\usepackage{amsmath,amssymb,amstext} 
\usepackage{fancyhdr}
\usepackage{float}
\usepackage{graphicx}
\setlength{\parindent}{0pt}
\setlength{\parskip}\medskipamount
\usepackage{multicol}
\usepackage{fancybox}
\usepackage{marvosym}
\usepackage[right]{eurosym}
\usepackage{geometry}
\usepackage{enumitem}
\usepackage{calc}
\usepackage{ifthen}
\newcounter{zettelnummer}

\newboolean{fertig}
\setboolean{fertig}{true}

\newcommand{\und}{}
\newcommand{\neturl}[3]{\url{#1}\ (Zugriff am #2; #3 Uhr)}
\newwrite\tempfile

\usepackage{atbegshi,picture}



\usepackage{hyperref}
\hypersetup{
    unicode=true,
    pdftoolbar=true,
    pdfmenubar=true,
    pdffitwindow=true,
    pdfstartview={FitH},
    pdfproducer={LaTeX2e},
    pdfnewwindow=false,
    colorlinks=true,
    linkcolor=black,
    citecolor=black,
    filecolor=black,
    urlcolor=black
}
\usepackage{url}
\geometry{a4paper, top=25mm, left=30mm, right=25mm, bottom=30mm, headsep=10mm, footskip=12mm}

%Diese Angaben müssen für jedes Modul geändert werden:
\newcommand{\theUni}{Universität}
\newcommand{\theOrt}{Oldenburg}
\newcommand{\theStudienfach}{Informatik}
\newcommand{\theModulName}{Modulname}
\newcommand{\theSemester}{SoSe2009}
\newcommand{\theTutorArt}{Tutor}% Tutorin, Dozent, Dozentin
\newcommand{\theTutor}{Peter Muster}
\newcommand{\theTutoriumArt}{Tutorium}%Tutorium oder Übung
\newcommand{\theGruppenmitglieder}{Steffen Nachname1\und Vorname Nachname2} %Es können mehrere Gruppenteilnehmer (durch \und getrennt) eingetragen werden. Vor \und kein Leerzeichen.
\newcommand{\OptionaleAngabe}{} %z.B. Gruppennummer oder Gruppenbuchstabe hier eintragen. a la Gruppe A
\newcommand{\theDayTut}{Do} %Hier Tag es Tutoriums eintragen. a la Mo, Di, Mi, Do, Fr
\newcommand{\theZeitBegin}{12} %Hier Anfangszeit des Tutoriums eingeben. a la 08, 09, 10 ... 18, 19 usw.
\newcommand{\theZeitEnde}{14} %Hier Endzeit des Tutoriums eingeben. a la 09, 10 ... 18, 19, 20 usw.
\newcommand{\theAbgabeZeit}{10:00} % Uhrzeit bis zu der die Abgabe gemacht werden muss


%Diese Angaben müssen für jeden Zettel geändert werden:
\setcounter{zettelnummer}{1}
\newcommand{\theAbgabeDatum}{15.05.12}

\begin{document}
\pagestyle{fancy}
\AtBeginShipout{\AtBeginShipoutUpperLeft{\put(\dimexpr\paperwidth-3mm\relax,-2cm){\makebox[0pt][r]{
\textbf{\Large %Diese Datei sollte nicht bearbeitet oder entfernt werden!
Zettel nicht fertig!
}}}}}

%Damit der Text besser zu lesen ist und die Silbentrennung nicht zickt:
\hbadness=4000
\tolerance=2000
\pretolerance=5000

\newcounter{aufgabenummer}
\newcounter{aufgabenteil}
\setcounter{aufgabenummer}{0}

\newcounter{subaufgabenummer}
\setcounter{subaufgabenummer}{0}

\newcommand{\aufgabe}[1]{\addtocounter{aufgabenummer}{1}\section*{\arabic{zettelnummer}.\arabic{aufgabenummer} #1}\setcounter{aufgabenteil}{0}\setcounter{subaufgabenummer}{0}}
\newcommand{\teil}{\addtocounter{aufgabenteil}{1}\subsection*{Aufgabenteil \alph{aufgabenteil}}\setcounter{subaufgabenummer}{0}}

\newlength{\platz}
\setlength{\platz}{\widthof{{\footnotesize Quelle:}}+6.252pt}
\newcommand{\subteil}{\paragraph*{\roman{subaufgabenummer}}\addtocounter{subaufgabenummer}{1}}
\newcommand{\quelle}[1]{\par{\footnotesize \textbf{Quelle:} \begin{minipage}[t]{\textwidth-\platz}\begin{flushleft}#1\end{flushleft}\end{minipage}}}

\renewcommand{\und}{\ \ }
\newlength{\ZeileA}
\newlength{\ZeileB}
\newlength{\ZeileC}
\newlength{\ZeileD}
\newlength{\ZeileE}
\setlength{\ZeileA}{\widthof{\textbf{\textsf{\theModulName{}}} \theSemester{}}}
\setlength{\ZeileB}{\widthof{\textbf{\textsf{Zettel \arabic{zettelnummer}}} Abgabe \theAbgabeDatum\ (\theAbgabeZeit\ Uhr)}}
\setlength{\ZeileC}{\widthof{\theTutoriumArt\ \theDayTut{}. \theZeitBegin\ -- \theZeitEnde\ Uhr bei \theTutor}}
\setlength{\ZeileD}{\widthof{\theGruppenmitglieder{}}}
\setlength{\ZeileE}{\widthof{\OptionaleAngabe{}}}

\newlength{\LaengsteZeile}
\setlength{\LaengsteZeile}{\ZeileA}
\ifdim\LaengsteZeile<\ZeileB
\setlength{\LaengsteZeile}{\ZeileB}
\fi
\ifdim\LaengsteZeile<\ZeileC
\setlength{\LaengsteZeile}{\ZeileC}
\fi
\ifdim\LaengsteZeile<\ZeileD
\setlength{\LaengsteZeile}{\ZeileD}
\fi
\ifdim\LaengsteZeile<\ZeileE
\setlength{\LaengsteZeile}{\ZeileE}
\fi

\renewcommand{\und}{\hfill }

\begin{center}
\begin{tabular}{|c|}
\begin{minipage}{\LaengsteZeile}
\begin{center}
\textbf{\textsf{\theModulName{}}} \hfill \theSemester{}\\
\textbf{\textsf{Zettel \arabic{zettelnummer}}} \hfill Abgabe \theAbgabeDatum\ (\theAbgabeZeit\ Uhr)\\
\theTutoriumArt\ \theDayTut{}. \theZeitBegin\ -- \theZeitEnde\ Uhr \hfill bei \theTutor\vspace*{.5mm} \\
\theGruppenmitglieder{}\\\OptionaleAngabe{}
\end{center}
\end{minipage}
\end{tabular} 
\end{center}

\newcounter{todo}
\setcounter{todo}{0}
\newcommand{\todo}[1]{\addtocounter{todo}{1}\begin{center}\shadowbox{\begin{minipage}{.7\textwidth}
\textbf{TODO \arabic{todo}:} #1
\end{minipage}}\end{center}\setboolean{fertig}{false}}

\newcommand{\tbd}{\marginpar{\begin{flushleft}\begin{footnotesize}
\fbox{Nicht fertig.}\end{footnotesize}
\end{flushleft}}\setboolean{fertig}{false}}

\newenvironment{frage}{\begin{small}\begin{itshape}}{\end{itshape}\end{small}\par}

\renewcommand{\und}{ \texorpdfstring{$\cdot$}{, } }
\fancyhead[L]{\theModulName{} (\theSemester{})}
\fancyhead[C]{}
\fancyhead[R]{\theTutorArt\ \theTutor{}}
\renewcommand{\headrulewidth}{0.4pt}
\fancyfoot[L]{{\small \theGruppenmitglieder{}}}
\fancyfoot[C]{}
\fancyfoot[R]{{\small Seite \thepage{}/\pageref{last}}}
\renewcommand{\footrulewidth}{0.0pt}


\hypersetup{
    pdftitle={\theModulName{} \theSemester{}},
    pdfkeywords={\theOrt{}, \theUni{}, \theStudienfach{}, \theModulName{}, \theSemester{}, \theAbgabeDatum, Zettel \arabic{zettelnummer}},
    pdfauthor={\theGruppenmitglieder{}},
    pdfsubject={Zettel \arabic{zettelnummer}, \theModulName{} \theSemester{}},
    pdfcreator={\OptionaleAngabe{}}
	}
\aufgabe{Name der Aufgabe}\tbd %Wenn eine Aufgabe den Befehl \tbd (to be done) enthält, so zeigt dies an, dass diese Aufgabe noch nicht fertig ist. Wenn eine Aufgabe vollständig bearbeitet ist, dann muss \tbd entfernt werden. Wenn keine Aufgabe mehr ein \tbd enthält, so verschwindet auch der Vermerk "Zettel nicht fertig!" auf allen Seiten.
\begin{frage}
Hier Aufgabentext (Optional)
\end{frage}

Hier Text
\quelle{
    Laudon, Kenneth C. and Laudon, Jane Price and Schoder, Detlef (2010). Wirtschaftsinformatik: Eine Einführung. Pearson Studium München,\\
    L. J. Heinrich, Dirk Stelzer: Informationsmanagement - Grundlagen, Aufgaben, Methoden. 9. Auflage. Oldenbourg Wissenschaftsverlag, München/Wien 2009
    }

\teil

Hier Text

\teil

Hier Text

\subteil

Hier Text

\subteil

Hier Text
\quelle{
    \neturl{http://www.uni-oldenburg.de/}{12.05.12}{21:15},\\
    Andrew S. Tanenbaum: Computernetzwerke. 5., aktualisierte Auflage, Pearson Studium, München 2012
}
\aufgabe{Name der Aufgabe}\tbd
\begin{frage}
Hier Aufgabentext (Optional)
\end{frage}
Hier Text

\teil
Hier Text 

\teil
Hier Text
%\aufgabe{Name der Aufgabe}\tbd 
\begin{frage}
Hier Aufgabentext (Optional)
\end{frage}
Hier Text

\teil
Hier Text


\teil
Hier Text

\subteil
Hier Text

\subteil
Hier Text
%\aufgabe{Name der Aufgabe}\tbd 
\begin{frage}
Hier Aufgabentext (Optional)
\end{frage}
Hier Text

\teil
Hier Text


\teil
Hier Text

\subteil
Hier Text

\subteil
Hier Text
\immediate\openout\tempfile=fertigtestfile.tex
\ifthenelse{\boolean{fertig}}{\immediate\write\tempfile{}}{\immediate\write\tempfile{Zettel nicht fertig!}}
\immediate\closeout\tempfile
\label{last}
\end{document}